\documentclass[review]{elsarticle}
\usepackage{microtype,blindtext,enumitem}
\usepackage[osf]{libertine}
\usepackage[english]{babel}

\makeatletter
\def\ps@pprintTitle{%
 \let\@oddhead\@empty
 \let\@evenhead\@empty
 \def\@oddfoot{}%
 \let\@evenfoot\@oddfoot}
\makeatother

\journal{Journal of CancerIQ}

\begin{document}

\begin{frontmatter}

\title{A method for evaluating Smith Syndrome cancer risk}

\author{Professor CancerIQ}
\address{541 N Fairbanks Ct, 22nd Floor, Chicago, IL 60611}
\ead[url]{www.canceriq.com}

\begin{abstract}
This is an entirely fake cancer risk model and should never be used for anything other than the CancerIQ coding challenge.
\end{abstract}

\end{frontmatter}

\section{Description of some research}

\blindtext[2]

\section{Our risk model}

\blindtext[1]

\subsection*{Testing criteria}
\begin{itemize}[noitemsep]

    \item An individual with a Smith Syndrome cancer\footnote{Smith Syndrome
    cancers include breast, colorectal, brain, endometrial, and kidney
    cancer.} meeting any of the following:

    \begin{itemize}[noitemsep, nolistsep]
        \item Smith Syndrome cancer diagnosed at or before age 60,
        \item An additional synchronous or metachronous Smith Syndrome
        cancer,
        \item One or more first-degree relatives with a Smith Syndrome
        cancer diagnosed at or before age 50, or
        \item Two or more first- or second-degree relatives with Smith
        Syndrome cancer or gastric cancer diagnosed at any age.

    \end{itemize}

    \item An individual with no personal history of a Smith Syndrome cancer,
        but with a close relative\footnote{Close relatives include first-,
        second-, or third-degree relatives.} with any of the following:

    \begin{itemize}[noitemsep, nolistsep]

        \item Two or more relatives with Smith Syndrome cancers, one diagnosed
        at or before age 50, or
        \item Male breast cancer.

    \end{itemize}

    \item A close relative meeting any of the above criteria.

\end{itemize}

\section{Some statistics and comparisons to other research}

\blindtext[2]

\end{document}

